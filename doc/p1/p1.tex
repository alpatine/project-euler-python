\documentclass{article}
\usepackage{mathtools}
\usepackage{pythontex}

\title{Project Euler 1 – Multiples of 3 or 5}
\author{Alister McKinley}
\date{}

\begin{document}
\maketitle

%\begin{abstract}
%The first problem of Project Euler involves summing numbers that are a multiple of 3 or 5.
%\end{abstract}

Project Euler Problem 1: https://projecteuler.net/problem=1

\section*{}
The problem reads:

\begin{quote}
    If we list all the natural numbers below 10 that are multiples of 3 or 5,
    we get 3, 5, 6 and 9. The sum of these multiples is 23. Find the sum of all
    the multiples of 3 or 5 below 1000.
\end{quote}

\section*{Brute Force Approach}
The most direct way to attack this problem is to test every number between 1
and 1000 to see if it is a multiple of 3 or a multiple of 5. If it is, add it
to a running total. After checking all of the numbers, the running total is the
answer. The following example is in Python.

\begin{pyverbatim}[][frame=single]
running_total = 0
for number in range(1000):
    if number % 3 == 0 or number % 5 == 0:
        running_total = running_total + number

print(running_total)
\end{pyverbatim}

Using more pythonic syntax, we can compress this to:

\begin{pyverbatim}[][frame=single]
print(sum(number for number in range(1000)
          if number % 3 == 0 or number % 5 == 0))
\end{pyverbatim}

Both of these will easily execute within the 1 minute timeframe, to give the
result of 233,168.

\section*{Using a formula}
We can define the sum of the first \(n\) multiples of \(k\) as the following:
\[ S_{k, n} = k + 2k + 3k + \dots + (n-2)k + (n-1)k + nk \]
Being a finite sum there is no problem with writing it in reverse order:
\[ S_{k, n} = nk + (n-1)k + (n-2)k + \dots + 3k + 2k + k \]
Note that \(S_{k,n}\) has exactly \(n\) terms. If we add both versions together
we get the following expression:
\[ \begin{array}{rlllllllllll}
    & S_{k, n} &= &k &+ &2k &+ &\cdots &+ &nk \\
    + & S_{k, n} &= &nk &+ &(n-1)k &+ &\cdots &+ &k \\
    \hline & 2S_{k, n} &= &(k+nk) &+ &(2k+(n-1)k) &+ &\cdots &+ &(nk+k) \\
    & &= &(k + nk) &+ &(k + nk) &+ &\cdots &+ &(nk + k) \\
    & &= &(n+1)k &+ &(n+1)k &+ &\cdots &+ &(n+1)k \\
    & &= &n(n+1)k \end{array}
\]

Dividing both sides by 2 we get a closed formula:
\[ S_{k, n} = \frac{n(n+1)k}{2} \]
For example, the sum of the first 4 multiples of 3 is:
\[ \begin{aligned}
    S_{3, 4} &= 3 + 6 + 9 + 12 \\
    &= 30
\end{aligned} \]
and using our formula:
\[ \begin{aligned}
    S_{3, 4} &= \frac{4 \cdot (4+1) \cdot 3}{2} \\
    &= \frac{60}{2} \\
    &= 30
\end{aligned} \]

It's tempting to jump in now and just add the sums of multiples of 3 or 5, but
this leads to a problem:
\[ \begin{aligned}
    \text{multiples of 3} &= \{3, 6, 9, 12, 15, 18, \dots, 987, 990, 993, 996,
                               999\} \\
    \text{multiples of 5} &= \{5, 10, 15, 20, 25, 30 \dots, 975, 980, 985, 990,
                               995\}
\end{aligned} \]

Note that 15 and 990 are in both sets. In fact, every multiple of 15 below 1000
is in both sets and will be counted twice if we add the sum of both sets
together, so we will need to subtract one copy of the sum of multiples of 15 to
correct the overcounting.

With that in mind, we now need a way to determine how many multiples to
include. We can do this simply by dividing 999 by our number in question and
then taking the floor function to round down to the nearest integer. We use 999
and not 1000 because we want multiples \emph{less than} 1000.
\[ \begin{aligned}
    \text{number of multiples of 3 less than 1000}
        &= \left \lfloor \frac{999}{3} \right \rfloor = 333 \\
    \text{number of multiples of 5 less than 1000}
        &= \left \lfloor \frac{999}{5} \right \rfloor = 199 \\
    \text{number of multiples of 15 less than 1000}
        &= \left \lfloor \frac{999}{15} \right \rfloor = 66 \\
\end{aligned} \]

With everything we have we are now ready to tackle this problem directly:
\[ \begin{aligned}
    \sum_{\substack{0 \leq n \leq 999\\3,5\mid n}}
    &= 3 + 5 + 6 + 9 + 10 + \dots + 995 + 996 + 999 \\
    &= \underbrace{(3 + 6 + \dots + 999)}_{\text{333 terms}}
     + \underbrace{(5 + 10 + \dots + 995)}_{\text{199 terms}}
     - \underbrace{(15 + 30 + \dots + 990)}_{\text{66 terms}} \\
    &= S_{3,333} + S_{5,199} - S_{15,66} \\
    &= \frac{333 \cdot (333+1) \cdot 3}{2}
        + \frac{199 \cdot (199+1) \cdot 5}{2}
        - \frac{66 \cdot (66+1) \cdot 15}{2} \\
    &= \frac{333666}{2} + \frac{199000}{2} - \frac{66330}{2} \\
    &= 166833 + 99500 - 33165 \\
    &= 233168
\end{aligned} \]

\end{document}