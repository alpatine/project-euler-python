\documentclass{article}
\usepackage{pythontex}
\usepackage{mathtools}
%\usepackage{amsfonts}
%\usepackage{amssymb}
%\usepackage{arydshln}

\DeclareMathOperator{\lcm}{lcm}

\title{Solutions to Project Euler Problem 5:\\Smallest Multiple}
\author{Alister McKinley}
\date{}

\begin{document}
\maketitle
\begin{center}
    https://projecteuler.net/problem=5
\end{center}

\section*{}
The problem reads:

\begin{quote}
    2520 is the smallest number that can be divided by each of the numbers from 1 to 10 without any remainder.

    What is the smallest positive number that is evenly divisible by all of the numbers from 1 to 20?
\end{quote}

\section*{Least Common Multiple}
Recall that the smallest number \(l\) that can be divided by two numbers \(a\), \(b\), can be referred to as the least common multiple is written as:
\[ \lcm(a,b)\]

There are several ways to calculate the least common multiple, but the approach we will explore is factorising each number into primes, then for each prime present included in \(a\) or \(b\) (or both), include the greatest power of that prime in \(l\).

Example: Find the \(\lcm\) of 90 and 420.
\[\begin{aligned}
    90 = 2 \cdot 3 \cdot 3 \cdot 5 &= 2^1 \cdot 3^2 \cdot 5^1 \cdot 7^0 \\
    420 = 2 \cdot 2 \cdot 3 \cdot 5 \cdot 7 &= 2^2 \cdot 3^1 \cdot 5^1 \cdot 7^1
\end{aligned}\]

Visual inspection of the prime factorisation shoes that the greatest power of 2 is \(2^2\), 3 is \(3^2\), 5 is \(5^1\), and 7 is \(7^1\). Multiplying these powers together gives the least common multiple:
\[ \lcm(90, 420) = 2^2 \cdot 3^2 \cdot 5^1 \cdot 7^1 = 1260\]

The least common multiple function can be extended to accecpt multiple numbers. To do this we factorise all of the numbers involved, and keep the highest prime powers just as we did for two numbers.

Example: Find the \(\lcm\) of 10, 15, and 20.
\[\begin{aligned}
    10 = 2 \cdot 5 &= 2^1 \cdot 3^0 \cdot 5^1 \\
    15 = 2 \cdot 3 \cdot 5 &= 2^1 \cdot 3^1 \cdot 5^1 \\
    20 = 2 \cdot 2 \cdot 5 &= 2^2 \cdot 3^0 \cdot 5^1
\end{aligned}\]

Again, we take the highest power of each prime and multiply them together to get the lcm:
\[\lcm(10, 15, 20) = 2^2 \cdot 3^1 \cdot 5^1 = 60\]

The problem can now be stated as finding \(\lcm(1, 2, 3, \dots, 20)\).

We can ignore the number 1 because adding it to the product will have no impact. So, we start by factoring the numbers from 2 to 20.
\[\begin{aligned}
    2 = 2 &= 2^1 \cdot 3^0 \cdot 5^0 \cdot 7^0 \cdot 11^0 \cdot 13^0 \cdot 17^0 \cdot 19^0 \\
    3 = 3 &= 2^0 \cdot 3^1 \cdot 5^0 \cdot 7^0 \cdot 11^0 \cdot 13^0 \cdot 17^0 \cdot 19^0 \\
    4 = 2 \cdot 2 &= 2^2 \cdot 3^0 \cdot 5^0 \cdot 7^0 \cdot 11^0 \cdot 13^0 \cdot 17^0 \cdot 19^0 \\
    5 = 5 &= 2^0 \cdot 3^0 \cdot 5^1 \cdot 7^0 \cdot 11^0 \cdot 13^0 \cdot 17^0 \cdot 19^0 \\
    6 = 2 \cdot 3 &= 2^1 \cdot 3^1 \cdot 5^0 \cdot 7^0 \cdot 11^0 \cdot 13^0 \cdot 17^0 \cdot 19^0 \\
    7 = 7 &= 2^0 \cdot 3^0 \cdot 5^0 \cdot 7^1 \cdot 11^0 \cdot 13^0 \cdot 17^0 \cdot 19^0 \\
    8 = 2 \cdot 2 \cdot 2 &= 2^3 \cdot 3^0 \cdot 5^0 \cdot 7^0 \cdot 11^0 \cdot 13^0 \cdot 17^0 \cdot 19^0 \\
    9 = 3 \cdot 3 &= 2^0 \cdot 3^2 \cdot 5^0 \cdot 7^0 \cdot 11^0 \cdot 13^0 \cdot 17^0 \cdot 19^0 \\
    10 = 2 \cdot 5 &= 2^1 \cdot 3^0 \cdot 5^1 \cdot 7^0 \cdot 11^0 \cdot 13^0 \cdot 17^0 \cdot 19^0 \\
    11 = 11 &= 2^0 \cdot 3^0 \cdot 5^0 \cdot 7^0 \cdot 11^1 \cdot 13^0 \cdot 17^0 \cdot 19^0 \\
    12 = 2 \cdot 2 \cdot 3 &= 2^2 \cdot 3^1 \cdot 5^0 \cdot 7^0 \cdot 11^0 \cdot 13^0 \cdot 17^0 \cdot 19^0 \\
    13 = 13 &= 2^0 \cdot 3^0 \cdot 5^0 \cdot 7^0 \cdot 11^0 \cdot 13^1 \cdot 17^0 \cdot 19^0 \\
    14 = 2 \cdot 7 &= 2^1 \cdot 3^0 \cdot 5^0 \cdot 7^1 \cdot 11^0 \cdot 13^0 \cdot 17^0 \cdot 19^0 \\
    15 = 3 \cdot 5 &= 2^0 \cdot 3^1 \cdot 5^1 \cdot 7^0 \cdot 11^0 \cdot 13^0 \cdot 17^0 \cdot 19^0 \\
    16 = 2 \cdot 2 \cdot 2 \cdot 2 &= 2^4 \cdot 3^0 \cdot 5^0 \cdot 7^0 \cdot 11^0 \cdot 13^0 \cdot 17^0 \cdot 19^0 \\
    17 = 17 &= 2^0 \cdot 3^0 \cdot 5^0 \cdot 7^0 \cdot 11^0 \cdot 13^0 \cdot 17^1 \cdot 19^0 \\
    18 = 2 \cdot 3 \cdot 3 &= 2^1 \cdot 3^2 \cdot 5^0 \cdot 7^0 \cdot 11^0 \cdot 13^0 \cdot 17^0 \cdot 19^0 \\
    19 = 19 &= 2^0 \cdot 3^0 \cdot 5^0 \cdot 7^0 \cdot 11^0 \cdot 13^0 \cdot 17^0 \cdot 19^1 \\
    20 = 2 \cdot 2 \cdot 5 &= 2^2 \cdot 3^0 \cdot 5^1 \cdot 7^0 \cdot 11^0 \cdot 13^0 \cdot 17^0 \cdot 19^0
\end{aligned}\]

Looking down the list we see that the answer is:
\[\begin{aligned}
    \lcm(1, 2, 3, \dots, 20) &= 2^4 \cdot 3^2 \cdot 5^1 \cdot 7^1 \cdot 11^1 \cdot 13^1 \cdot 17^1 \cdot 19^1 \\
    &= 16 \cdot 9 \cdot 5 \cdot 7 \cdot 11 \cdot 13 \cdot 17 \cdot 19 \\
    &= 232792560
\end{aligned}  \]

\section*{Using Logarithms}
Looking at the lcm calculation above, there was a lot of work put into factoring the 20 numbers. Here we will explore an alternative approach that can be used when looking for the lcm of consecutive numbers from 1 to \(n\).

Consider the highest power of 2 that was included: \(2^4 = 16\). That is the largest power of 2 less than 20 because \(2^5 = 32 > 20\). Now consider the highest power of 3 that was included: \(3^2 = 9\). That is the largest power of 3 less than 20 because \(3^3 = 27 > 20\). This pattern continues for all of the primes below 20.

This approach directly accounts for the lcm of the numbers 2, 3, 4, 5, 7, 8, 9, 11, 13, 16, 17, and 19 because they are all powers of a single prime less than the max power of that prime less than 20. What about the remaining numbers 6, 10, 12, 14, 15, 18, 20?

It turns out that all of those missing numbers are each a multiple of a combination of the already accounted for numbers. We have already shown there can be no higher power of any individual prime, the divisors a missing number must be less than the missing number, and every number has a unique prime factorisation. These three facts together mean that all of the missing numbers were actually accounted for as well.

Taking 20, the last missing number, as an example. It's prime factorisation is \(2^2 \cdot 5^1\). When we evaluate the powers of the primes this becomes \(4 \cdot 5\), both of which are on the already accounted for list.

So how do we quickly determine the highest power of a prime less than \(n\)? We use logarithms.

Exploring with the prime 2, we would expect the highest power to be 4, giving us the number 16 to include in the final product:
\[\begin{aligned}
    20 &= 20 \\
    \log_2{20} &= 4.321928095\dots && \text{(take base 2 log)}\\
    \lfloor\log_2{20}\rfloor &= 4 && \text{(take floor of both sides)} \\
    2^{\lfloor\log_2{20}\rfloor} &= 16 && \text{(exponentiate both sides with base 2)}
\end{aligned}\]

This approach works for each prime, so in general for an lcm from 1 to \(n\), we have:
\[\lcm(1, 2, 3, \dots, n) = \prod_{\substack{p \leq n\\p\text{-prime}}}{p^{\lfloor \log_p{n} \rfloor}}\]

In the case of \(\lcm(1, 2, 3, \dots, 20)\) we have:
\[\begin{aligned}
    \lcm(1, 2, 3, \dots, 20) ={}& \prod_{\substack{p \leq 20\\p\text{-prime}}}{p^{\lfloor \log_p{20} \rfloor}} \\
    ={}& 2^{\lfloor \log_2{20} \rfloor} \cdot 3^{\lfloor \log_3{20} \rfloor} \cdot 5^{\lfloor \log_5{20} \rfloor} \cdot 7^{\lfloor \log_7{20} \rfloor} \cdot 11^{\lfloor \log_{11}{20} \rfloor} \\
    {}& \cdot 13^{\lfloor \log_{13}{20} \rfloor} \cdot 17^{\lfloor \log_{17}{20} \rfloor} \cdot 19^{\lfloor \log_{19}{20} \rfloor} \\
    ={}& 2^4 \cdot 3^2 \cdot 5^1 \cdot 7^1 \cdot 11^1 \cdot 13^1 \cdot 17^1 \cdot 19^1 \\
    ={}& 232792560
\end{aligned}\]

\subsection*{Python Implementation}
\begin{pyverbatim}[][frame=single]
from math import floor, log, prod

primes = [2, 3, 5, 7, 11, 13, 17, 19]
print(prod(prime ** floor(log(20, prime)) for prime in primes))
\end{pyverbatim}

\end{document}