\documentclass{article}
\usepackage{mathtools}
\usepackage{amsfonts}
\usepackage{amssymb}
\usepackage{pythontex}
%\usepackage{arydshln}

\title{Solutions to Project Euler Problem 3:\\Largest Prime Factor}
\author{Alister McKinley}
\date{}

\begin{document}
\maketitle
\begin{center}
    https://projecteuler.net/problem=3
\end{center}

\section*{}
The problem reads:

\begin{quote}
    The prime factors of 13195 are 5, 7, 13 and 29.

    What is the largest prime factor of the number 600851475143 ?
\end{quote}

\section*{Dividing Out Approach}
The Fundamental Theorem of Arithmetic states that any integer greater than 1 can be expressed as a unique product of prime factors. This product can be written in a canonical form as follows:
\[ n = p_1^{r_1}p_2^{r_2}p_3^{r_3} \dots p_n^{r_n} \]
\[ \begin{aligned}
    & \text{where:} \\
    & n, r_i \in \mathbb{Z}_{\geq 1} && \text{is a positive integer} \\
    & p_1 < p_2 < \dots < p_i \in \mathbb{N} && \text{is prime}
\end{aligned} \]

Our dividing out approach will repeatedly divide \(n\) by an increasing denominator \(d\) in order to discover the prime factors of \(n\). Once this is complete, the largest prime factor found is the solution to the problem.

We start by setting our denominator \(d\) equal to the smallest prime:
\[d \leftarrow 2\]

Next, if \(d\) divides \(n\) we assign \(n\) the value of \(\frac{n}{d}\) and we assign the highest factor found so far, \(p_\text{max}\), the value of \(d\). If \(d\) does not divide \(n\), we increment \(d\) by 1.

\[\begin{aligned}
    & \text{If \(d \mid n\):} &&  n \leftarrow \frac{n}{d}  & && & \text{If \(d \nmid n\):} && d \leftarrow d + 1 \\
    & && p_\text{max} \leftarrow d
\end{aligned}\]

This process continues until \(d\) is larger than \(n\). Once this occurs, \(p_\text{max}\) contains the answer to the problem.

The following table will demonstrate this process for the number 13195. As per the example we expect the highest prime factor to be 29.

\[\begin{array}{|c|c|c|c|c|}
    \hline
    n & d & n/d \in \mathbb{Z} & n/d & p_\text{max} \\
    \hline
    13195 & 2 & \text{False} & & \\
    13195 & 3 & \text{False} & & \\
    13195 & 4 & \text{False} & & \\
    13195 & 5 & \text{True} & 2639 & 5 \\
    2639 & 5 & \text{False} & & 5 \\
    2639 & 6 & \text{False} & & 5 \\
    2639 & 7 & \text{True} & 377 & 7 \\
    377 & 7 & \text{False} & & 7 \\
    377 & 8 & \text{False} & & 7 \\
    377 & 9 & \text{False} & & 7 \\
    377 & 10 & \text{False} & & 7 \\
    377 & 11 & \text{False} & & 7 \\
    377 & 12 & \text{False} & & 7 \\
    377 & 13 & \text{True} & 29 & 13 \\
    29 & 13 & \text{False} & & 13 \\
    29 & 14 & \text{False} & & 13 \\
    29 & 15 & \text{False} & & 13 \\
    29 & 16 & \text{False} & & 13 \\
    29 & 17 & \text{False} & & 13 \\
    29 & 18 & \text{False} & & 13 \\
    29 & 19 & \text{False} & & 13 \\
    29 & 20 & \text{False} & & 13 \\
    29 & 21 & \text{False} & & 13 \\
    29 & 22 & \text{False} & & 13 \\
    29 & 23 & \text{False} & & 13 \\
    29 & 24 & \text{False} & & 13 \\
    29 & 25 & \text{False} & & 13 \\
    29 & 26 & \text{False} & & 13 \\
    29 & 27 & \text{False} & & 13 \\
    29 & 28 & \text{False} & & 13 \\
    29 & 29 & \text{True} & 1 & 29 \\
    \hline
\end{array}\]

We can shorten this process by considering the maximum value of the denominator we will use when dividing. We will set this maximum value as follows:
\[ d_\text{max} = \left\lfloor\sqrt{n}\right\rfloor\]
The process now ends as soon as we are attempting to divide by a factor that is larger than this calculated maximum i.e. \(d > d_\text{max}\). At this point the number we are attempting to divide, \(n\), is the largest prime factor we will find. The process becomes:

\[\begin{aligned}
    & \text{If \(d \mid n\):} &&  n \leftarrow \frac{n}{d}  & && & \text{If \(d \nmid n\):} && d \leftarrow d + 1 \\
    & && d_\text{max} \leftarrow \left\lfloor\sqrt{n}\right\rfloor 
\end{aligned}\]

The following table will demonstrate this shortened process for the number 13195. As per the example we expect the highest prime factor to be 29.

\[\begin{array}{|c|c|c|c|c|}
    \hline
    n & d_\text{max} & d & n/d \in \mathbb{Z} & n/d \\
    \hline
    13195 & 114 & 2 & \text{False} & \\
    13195 & 114 & 3 & \text{False} & \\
    13195 & 114 & 4 & \text{False} & \\
    13195 & 114 & 5 & \text{True} & 2639 \\
    2639 & 51 & 5 & \text{False} & \\
    2639 & 51 & 6 & \text{False} & \\
    2639 & 51 & 7 & \text{True} & 377 \\
    377 & 19 & 7 & \text{False} & \\
    377 & 19 & 8 & \text{False} & \\
    377 & 19 & 9 & \text{False} & \\
    377 & 19 & 10 & \text{False} & \\
    377 & 19 & 11 & \text{False} & \\
    377 & 19 & 12 & \text{False} & \\
    377 & 19 & 13 & \text{True} & 29 \\
    29 & 5 & 13 & & \\
    \hline
\end{array}\]

The process ended when \(d\) became larger than \(d_\text{max}\). As expected the number we were dividing, \(n=29\), is the largest prime factor.

\subsection*{Python Implementation}

\begin{pyverbatim}[][frame=single]
def prime_factors(number: int) -> Set[int]:
    """Calculate the primes that will divide number.

    Prime numbers with a power of zero are not included.
    Prime numbers with a power greater than 1 are not repeated.
    """
    result = set()
    if number <= 1:
        return result
    current_numerator = number
    current_factor = 2
    max_factor = floor(sqrt(number))
    while current_factor <= max_factor:
        quotient, remainder = divmod(current_numerator,
                                     current_factor)
        if remainder == 0:
            current_numerator = quotient
            max_factor = floor(sqrt(current_numerator))
            result.add(current_factor)
        else:
            current_factor = current_factor + 1
    result.add(current_numerator)
    return result

print(max(prime_factors(number)))
\end{pyverbatim}


\end{document}