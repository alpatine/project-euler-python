\documentclass{article}
\usepackage{pythontex}
\usepackage{mathtools}
%\usepackage{amsfonts}
%\usepackage{amssymb}
%\usepackage{arydshln}

\title{Solutions to Project Euler Problem 6:\\Sum Square Difference}
\author{Alister McKinley}
\date{}

\begin{document}
\maketitle
\begin{center}
    https://projecteuler.net/problem=6
\end{center}

\section*{}
The problem reads:

\begin{quote}
    The sum of the squares of the first ten natural numbers is,
    \[1^2+2^2+\dots+10^2 = 385\]

    The square of the sum of the first ten natural numbers is,
    \[(1+2+\dots+10)^2 = 55^2 = 3025\]

    Hence the difference between the sum of the squares of the first ten natural numbers and the square of the sum is \(3025-385=2640\).

    Find the difference between the sum of the squares of the first one hundred natural numbers and the square of the sum.
\end{quote}

\section*{Brute Force}
This answer can be obtained by simply performing the artihmetic, although there are many calculations that each present an opportunity for a mistake. Doing this we get:
\[\begin{aligned}
    (1+2+\dots+99+100)^2 = 5050^2 &= 25502500 \\
    1^2 + 2^2 + \dots + 99^2 + 100^2 = 1 + 4 + \dots + 9801 + 10000 &= 338350 \\
    25502500 - 338350 &= 25164150
\end{aligned}\]

\pagebreak
\subsection*{Python Implementation}
\begin{pyverbatim}[][frame=single]
def sum_numbers(n: int) -> int:
    return n * (n + 1) // 2

def sum_square_numbers(n: int) -> int:
    return n * (2 * n + 1) * (n + 1) // 6

def p6(n: int) -> int:
    return sum_numbers(n) ** 2 - sum_square_numbers(n)

print(p6(100))
\end{pyverbatim}

\section*{Using Algebra}
To reduce the potential for errors, we will derive a formula that can be used to calculate the result directly.

The triangle numbers are a well known sequence formed by adding the first n integers:
\[\begin{aligned}
    T_1 &= 1 \\
    T_2 &= 1 + 2 = 3 \\
    T_3 &= 1 + 2 + 3 = 6 \\
    \vdots
\end{aligned}\]

The formula for a given triangle number is also well known:
\[T_n = \sum_{k=1}^{n}k = \frac{n}{2}(n+1)\]

Previously we examined the sum of squares sequence and determined that the formula for a given term in that sequence is:
\[\begin{aligned}
    S_n = \sum_{k=1}^{n}k^2 = \frac{n}{6}(n+1)(2n+1)
\end{aligned}\]

Combining these formulae together will allow us to derive a new formula that we can use to calculate the desired result:
\[\begin{aligned}
    \text{Let: } A_n &= (1+2+\dots+n)^2 - (1^2 + 2^2 + \dots + n^2) \\
    &= \left(\sum_{k=1}^{n}k\right)^2 - \left(\sum_{k=1}^{n}k^2\right) \\
    &= \left(\frac{n}{2}(n+1)\right)^2 - \frac{1}{6}n(n+1)(2n+1) \\
    &= \frac{n^2}{4}(n+1)^2 - \frac{1}{6}n(n+1)(2n+1) \\
    &= \frac{n}{12}\left(3n(n^2+2n+1)-2(2n^2+3n+1)\right) \\
    &= \frac{n}{12}\left(3n^3 + 2n^2 - 3n - 2\right) \\
    &= \frac{n}{12}(n-1)(n+1)(3n+2)
\end{aligned}\]

The question is asking for the 100th term. Plugging that in gives the desired result:
\[\begin{aligned}
    A_{100} &= \frac{100}{12}(100-1)(100+1)(3\cdot100+2) \\
    &= \frac{100}{12}\cdot99\cdot101\cdot302 \\
    &= 25164150
\end{aligned}\]

\end{document}